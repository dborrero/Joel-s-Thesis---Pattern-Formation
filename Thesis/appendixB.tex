\chapter{Code}

\section{Gray-Scott simulation code} \label{appB:gs-code}

The following code is written in the Python language. It requires the Numpy package which allows for easy and intuitive manipulation of matrices as well as Matplotlib to display the simulation. The procedure \textsc{runGS} takes four arguments: $d_u$, $d_v$, $F$, and $k$. By default, the domain size $n$ is set to 256.

\begin{Verbatim}[fontsize=\footnotesize,frame=leftline,framesep=5mm]
import numpy as np
import matplotlib.pyplot as plt

def runGS(Du, Dv, F, k):
  n = 256
  
  # create a structured n+2 by n+2 array of double precision floats
  Z = np.zeros((n+2,n+2), [('U', np.double), ('V', np.double)])
  U,V = Z['U'], Z['V']
  
  # u, v represent the concentrations of U, V
  u,v = U[1:-1,1:-1], V[1:-1,1:-1]
  
  # set initial conditions
  r = 20
  u[...] = 1.0  # set all u to 1.0
  U[n/2-r:n/2+r,n/2-r:n/2+r] = 0.50
  V[n/2-r:n/2+r,n/2-r:n/2+r] = 0.25
  
  # 'sprinkling' of random noise
  u += 0.05*np.random.random((n,n))
  v += 0.05*np.random.random((n,n))

  # set up plot
  plt.ion()

  # plot options
  size = np.array(Z.shape)
  dpi = 120.0
  figsize= size[1]/float(dpi),size[0]/float(dpi)
  fig = plt.figure(figsize=figsize, dpi=dpi, facecolor="white")
  fig.add_axes([0.0, 0.0, 1.0, 1.0], frameon=False)
  cmap = plt.cm.binary  # this is a greyscale colormap
  im = plt.imshow(V, interpolation='bicubic', cmap=cmap)  # show V in the plot
  plt.xticks([]), plt.yticks([])	

  # run simulation for 25000 time steps
  for i in xrange(25000):
    # discretized Laplacian matrix for u
    Lu = ( U[0:-2,1:-1] + U[2: ,1:-1] + 
       U[1:-1,0:-2] + U[1:-1,2: ] - 4*U[1:-1,1:-1] )
       
    # discretized Laplacian matrix for v
    Lv = ( V[0:-2,1:-1] + V[2: ,1:-1] + 
       V[1:-1,0:-2] + V[1:-1,2: ] - 4*V[1:-1,1:-1] )

    uvv = u*v*v  # the nonlinear term uv^2
      
    # change the concentrations in place
    u += (Du*Lu - uvv +  F   *(1-u))
    v += (Dv*Lv + uvv - (F+k)*v    )

    if i % 10 == 0:  # show only every 10 steps on the plot
      im.set_data(V)
      im.set_clim(vmin=0.0, vmax=0.4)  # set color limits
      plt.draw()
      # to save each figure
      plt.savefig('./gs/gs-%04d.png' % (i/10) ,dpi=dpi)

  plt.ioff()
  plt.close()
\end{Verbatim}

\section{Entropy calculation code}

The code below is used to calculate the entropy maps shown in \refFigs{fig:V144_fk_rs_entropy}{fig:UV_entropy_trans}. It is also written in the Python language and relies on a few extra packages. The procedure \textsc{bettiList} by examining a pairs of Betti numbers contained in a CSV file with rows of the form \texttt{time-step,b0,b1} where \texttt{b0,b1} indicate $\beta_0$ and $\beta_1$. \textsc{makeP\_i} uses this list to count the number of times each pair \verb+{b0,b1}+ (interpreted simply as a string) occurs in the CSV (like a histogram). This list is normalized by the total number of pairs (equal to the number of time steps) to give a list $P_i$ for each pair. The procedure \textsc{entropy} takes the list of $P_i$ and returns the entropy for the system, $S$.

\begin{Verbatim}[fontsize=\footnotesize,frame=leftline,framesep=5mm]
import os
import subprocess
import numpy as np
from collections import Counter
import csv

# calculates entropy given a list of P_i
def entropy(P_i):
  S = 0.0
  for i in P_i:
    S += -(i*np.log(i)) # the natural log
  return S

# creates a list of all Betti number pairs {b0, b1} given a CSV file
def bettiList(csvfile):
  betti = []
  with open(csvfile, 'rU') as file:
    reader = csv.reader(file, delimiter=',')
    for row in reader:
      b0b1 = row[1: 3] # convert the b0, b1 part to a string like '[b0 b1]'
      b0b1str = ','.join(b0b1)
      betti.append(b0b1str)
  file.close()
  return betti

# makes a list of the probability of each state, P_i
def makeP_i(csvfile):
  betti = bettiList(csvfile)
  N = len(betti)
  hist = Counter(betti).items() # counts how many times each state occurs
  hist.sort(lambda x, y: cmp(x[1], y[1]), reverse=True) # largest P first
  # normalize
  P_i = np.array([np.divide(pair[1], N, dtype=np.float) for pair in hist])
  return P_i, hist, N

# saveEntropyCSV creates a CSV of entropies for each F, k pair
# input is a folder of CSVs (one for each F, k pair)
# containing {b0, b1} at each time step
def saveEntropyCSV(infolder, outfile):
  filelist = os.listdir(infolder)
  subprocess.call(['touch',outfile])
  csv = open(outfile, 'r+')
  for csvfile in filelist:
    P_i = makeP_i( infolder + '/' + csvfile )[0]
    S = entropy(P_i)
    F = csvfile.split('_')[0] # e.g. '0.044_0.038.csv'
    k = csvfile.split('_')[1][ :-4] # remove '.csv'
    csv.write( F + ',' + k + ',' + str(S) + '\n') 
  csv.close()
\end{Verbatim}