\chapter{Reaction-Diffusion Systems}

	Reaction-diffusion (RD) systems are merely models that determine how concentrations of substances change in space. These systems are driven by two processes: chemical reaction and spatial diffusion. RD systems are partial differential equations, the most basic of which might look something like \refeq{eq:KPP}. This is sometimes called the Kolmogorov-Petrovsky-Piskounov equation in which $u$ is a generic chemical species, $d$ is a diffusion coefficient, $\nabla^2 u$ is the Laplace operator, and $r(u)$ is a general reaction term.
	\begin{align}
		\frac{\partial u}{\partial t} = d \nabla^2 u + r(u)
		\label{eq:KPP}
	\end{align}
What makes RD systems interesting, however, is the wide variety of patterns they form and how many of those patterns resemble patterns of nature such as spirals, stripes, and spots. In 1952, Alan Turing suggested that RD systems of morphogens may be able to explain the presence of spots or stripes on an organism \rf{turing_1952}. Although the science behind animal patterns is more complicated, Turing laid the framework by which patterns form from minor perturbations of otherwise homogenous systems. Since then, many others have noted the similarity between RD patterns and patterns in nature\footnote{Among others, Bard, 1974 or 1981; Murray, 1981; Meinhardt, 1982; Young, 1984; Meinhardt and Klinger, 1987; or Turk, 1991.}.
	
	
\section{The Gray-Scott Model}
	The Gray-Scott system models the reaction of two generic chemical species, $U$ and $V$, whose concentration in space is represented by $u$ and $v$ respectively. The model is based on the chemical reaction in \refeq{eq:gs-chem}.
	\begin{align}
	\begin{split}
		U + 2V &\rightarrow 3V \\
		V &\rightarrow P
		\label{eq:gs-chem}
	\end{split}
	\end{align}
$V$ is converted to inert product, $P$, which doesn't interfere with the reaction of the system. $V$ appears on both sides of the chemical reaction and thus catalyzes its own production. Gray, Scott developed the following non-dimensional PDE:
	\begin{align}
		\frac{\partial u}{\partial t} & = d_u \nabla^2 u - uv^2 + f(1-u) \\
		\frac{\partial v}{\partial t} & = d_v \nabla^2 v  + uv^2 - (f +k)v
		\label{eq:gs}
	\end{align}
We see that both equations take the form of \refeq{eq:KPP} only they are coupled. For simplicity, $d_u$, $d_v$, $f$, and $k$ are taken to be constants. The first term in each equation, $d_u \nabla^2$ and $d_v \nabla^2$, are the diffusion terms. The Laplace operator, $\nabla^2$, is responsible for the diffusion of each chemical in space (like the diffusion of heat in the more familiar heat equation) while the \textit{diffusion coefficients}, $d_u$ and $d_v$, govern the rate. The $\pm uv^2$ term is the  \textit{reaction term} which converts $U$ into $V$: an increase in $v$ is equal to the decrease in $u$, hence $+uv^2$ in the second equation. Since $U$ will eventually get used up to generate $V$, the term $f(1-u)$ is the \textit{replenishment term} which reintroduces chemical $U$ into the system ($u$ has a maximum value of 1). Similarly, chemical $V$ would increase without limit except for the \textit{diminishment term}, $(f+k)v$, which serves to remove chemical $V$ from the system. One can imagine this system as two chemicals interacting in a cell where the bloodstream carries chemicals in and out.
	
