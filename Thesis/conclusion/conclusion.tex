\chapter*{Conclusion}
         \addcontentsline{toc}{chapter}{Conclusion}
	\chaptermark{Conclusion}
	\markboth{Conclusion}{Conclusion}
	\setcounter{chapter}{4}
	\setcounter{section}{0}
	
Throughout this thesis, we have tread the territory of nonlinear dynamics, computation, topology, and combinatorics. We have shown how the combination of these theories can produce novel and compelling results when presented with complex information. By examining patterns at their lowest level, looking closely at the fundamental geometric structures that make up the dazzling images we see, we can extract meaningful information and elucidate their properties. In this case, the analysis of the Gray-Scott system led us to calculate the entropy of the system over a wide range of parameters which both complements and confirms the analysis derived directly from the physics of the system.

The homology theory presented here has given us interesting insight into the dynamics of at least one pattern-forming model, but as I have endeavored to show, one can extend these techniques to \textit{any} topological object. This may be a single image, a video of an experiment, or a 4D construction of medical imaging data; the tools of homology are amazingly resilient and as computational methods evolve, these techniques may take precedence in the field of image analysis.

Of course, the findings in this thesis point towards many more avenues for further investigation of homology. One of the major problems confronted in this thesis is finding an appropriate threshold for which to perform the Betti number calculations. This could be solved with the implementation of an adaptive thresholding algorithm or, with a large leap in complexity, applying persistent homology (a relatively young theory at this time). Another interesting extension would be to use the time series information to derive more telling mathematical quantities such as the Lyapunov exponent which would confirm the chaotic dynamics of a system. It is certain that the applications of homology theory have not yet been exhausted; one of the virtues is that homology is so fundamental, its applications are wide open. It is my hope that the reader is convinced of its usefulness in the wake of increasingly complex data.