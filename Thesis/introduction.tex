%The \introduction command is provided as a convenience.
%if you want special chapter formatting, you'll probably want to avoid using it altogether
		
\chapter*{Introduction}
    \addcontentsline{toc}{chapter}{Introduction}
		\chaptermark{Introduction}
		\markboth{Introduction}{Introduction}
% The three lines above are to make sure that the headers are right, that the intro gets included in the table of contents, and that it doesn't get numbered 1 so that chapter one is 1.

	The study of pattern formation is incredibly diverse and certainly one of the most compelling aspects of nonlinear phenomenology. Scientists from many disciplines have analyzes patterns on the scale of the entire universe all the way down to the microscopic\footnote{See the introduction of Cross \& Greenside for an overview of the study of pattern formation.}. Just a cursory glance at the structure of a wind-swept sand dune, a snowflake, or even our own galaxy reveals something interesting. Observation of these patterns might lead a scientist to ask what causes the pattern and wonder why there are patterns at all. This question gets complicated quickly because whether it's God in the patterns or the result of a non-equilibrium universe, there is still the question of what it means to have ``structure'' or ``complexity'' or to be ``interesting''.
	
	Fortunately, the theoretical model of pattern formation discussed here allows me to sidestep all of these questions while focusing primarily on the method of analysis. There are many mathematical tools available to help understand pattern formation but applying new methods and techniques is essential, especially when analyzing a tired system like the Gray-Scott model.