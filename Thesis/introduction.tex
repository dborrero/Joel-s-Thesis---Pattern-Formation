%The \introduction command is provided as a convenience.
%if you want special chapter formatting, you'll probably want to avoid using it altogether
		
\chapter*{Introduction}
    \addcontentsline{toc}{chapter}{Introduction}
		\chaptermark{Introduction}
		\markboth{Introduction}{Introduction}
% The three lines above are to make sure that the headers are right, that the intro gets included in the table of contents, and that it doesn't get numbered 1 so that chapter one is 1.

	The study of pattern formation is incredibly diverse. Scientists from many disciplines have analyzes patterns on the scale of the entire universe all the way down to the microscopic\footnote{See the introduction of Cross \& Greenside for an overview of the study of pattern formation.}. Just a cursory glance at the structure of a wind-swept sand dune, a snowflake, or even our own galaxy reveals something interesting. Observation of these patterns might lead a scientist to ask what causes the pattern and wonder why there are patterns at all. This question gets complicated quickly because whether it's God in the patterns or the result of a non-equilibrium universe, there is still the question of what it means to have ``structure'' or ``complexity'' or to be ``interesting''.
	
	Fortunately, the theoretical model of pattern formation discussed here allows me to sidestep all of these questions while focusing primarily on the method of analysis. There are many mathematical tools available to help understand pattern formation but applying new methods and techniques is essential, especially when analyzing a tired system like the Gray-Scott model.
	
	Welcome to the \LaTeX\ thesis template. If you've never used \TeX\ or \LaTeX\ before, you'll have an initial learning period to go through, but the results of a nicely formatted thesis are worth it for more than the aesthetic benefit: markup like \LaTeX\ is more consistent than the output of a word processor, much less prone to corruption or crashing and the resulting file is smaller than a Word file. While you may have never had problems using Word in the past, your thesis is going to be about twice as large and complex as anything you've written before, taxing Word's capabilities. If you're still on the fence about  using \LaTeX, read the Introduction to LaTeX on the CUS site as well as skim the following template and give it a few weeks. Pretty soon all the markup gibberish will become second nature.

\section{Why use it?}
	
\LaTeX\ does a great job of formatting tables and paragraphs. Its line-breaking algorithm was the subject of a PhD.\thinspace thesis. It does a fine job of automatically inserting ligatures, and to top it all off it is the only way to typeset good-looking mathematics.

\section{Who should use it?}

Anyone who needs to use math, tables, a lot of figures, complex cross-references, IPA or who just cares about the final appearance of their document should use \LaTeX. At Reed, math majors are required to use it, most physics majors will want to use it, and many other science majors may want it also.